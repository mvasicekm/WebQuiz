\documentclass[language = czech]{webquiz}

\BreadCrumbs{Matematika https://www.math.muni.cz/\string~xvasicekm | Kombinatorika https://www.math.muni.cz/\string~xvasicekm/kombiIndex.html | title}

\InstitutionURL{https://www.math.muni.cz}

\title{Variace a Permutace}

\usepackage[czech]{babel}
\usepackage{hyperref}
\usepackage[dvipdfmx]{graphicx}

\begin{document}

	\begin{discussion}[Variace][Variace]\label{d1} %Prometheus 
		$k$-\textbf{členná variace} z $n$ prvků je uspořádaná $k$-tice sestavená z těchto prvků tak, že každý se v ní vyskytuje nejvýše jednou.\\
		Počet $V(k,n)$ všech $k$-členných variací z $n$ prvků je \[V(k,n)=n(n-1)(n-2)\dots(n-k+1) = \dots = \frac{n!}{(n-k)!}.\]%poslední úprava je z Poláka.
	\end{discussion}
	
	\begin{discussion}[Permutace][Permutace]\label{d2} %Prometheus
		 \textbf{Permutace} z $n$ prvků je každá $n$-členná variace z těchto prvků tak, že každý se v ní vyskytuje právě jednou.\\
		 Počet $P(n)$ všech permutací z $n$ prvků je \[P(n)=n!.\]
	\end{discussion}
	
	\begin{question} \label{o1} %7.1-53/83
		Před odjezdem na prázdniny si 8 kamarádů slíbilo, že si navzájem pošlou pohlednice z cest. Kolik pohlednic si celkem poslali, jestliže všichni splnili svůj slib? 	
		\answer[integer]{56}
		\whenRight Přesně tak :)  \qref[Další otázka]{o2}
		\whenWrong Pojďme se na to podívat:\\ Pokud vybereme libovolnou dvojici kamarádů, tak si vzájemně poslali pohlednici. V řeči matematiky můžeme říct, že tvoříme uspořádané dvojice z 8 prvků (kamarádů). Tedy můžeme použít \dref[variace]{d2} $V(2,8)$.
	\end{question}
	
	\begin{question} \label{o2} %7.1-59/83
		Určete počet všech přirozených čísel, která jsou (v dekadickém zápisu):
		\begin{enumerate}
			\item dvojciferná
			\item trojciferná
			\item čtyřciferná
		\end{enumerate} a mají vesměs různé číslice. 
		\begin{choice}[columns=2]
			\incorrect 1. 90, 2. 720, 3. 5040.
				\feedback Nezapomeň, že se nepočítají dvojice, trojice nebo čtveřice, které začínají nulou (např.: 0123 = 123, tedy není čtyřciferné číslo).
			\incorrect 1. 81, 2. 729, 3. 6561.
				\feedback Čísla mají obsahovat různé číslice, tedy třeba číslo 11 nebo 2215 nevyhovuje zadání.
			\incorrect 1. 72, 2. 504, 3. 3024. 
				\feedback Zkus si rozmyslet, z jakých číslic tvoříš čísla a kolik mají cifer. Potom můžeš použít \dref[variace]{d2}. Nebo je možné použít \href{https://www.youtube.com/watch?v=JLheSkk4yWQ}{pravidlo součinu}
			\correct 1. 81, 2. 648, 3. 4536.
				\feedback Dobrá práce \qref[další otázka]{o3} už čeká :)
		\end{choice}	
	\end{question}
		
	\begin{question} \label{o3} %7.1-64/83
		Z kolika prvků lze vytvořit právě 420 různých dvojčlenných variací
		\begin{choice}[columns=2]
			\incorrect $21$ nebo $-20$ prvků
			\feedback $-20$ prvků úloze nevyhovuje.
			\correct $21$
			\feedback Správně! Můžeš pokračovat na \qref[další otázku]{o4}.
			\incorrect $42$
			\feedback Zkus si sestavit rovnici, když víš, že $V(2,n)=420$. Pokud si ani tak nevíš rady, zkus se podívat na \dref[variace]{d1}.
			\incorrect Žádná z uvedených možností
			\feedback Bohužel, řešení ve výběru je. Zkoušel(a) sis sestavit rovnici?
		\end{choice}
	\end{question}
	
	\begin{question} \label{o4} %7.1-65a)/83
		Řešte v oboru $\mathbb{N}$ rovnici s neznámou $n$. \[V(3,n)=12\cdot V(2,n).\]
		\answer[integer]{14}
		\whenRight Správně, \qref[další otázka]{o5}.
		\whenWrong Zkus si rovnici rozepsat podle \dref[variací]{d1}
	\end{question}
	
	\begin{question} \label{o5} %7.1-67/84
		Kolika různými způsoby lze rozsadit čtyři osoby na čtyři různá místa?
		\begin{choice}[multiple]
			\incorrect Pravidlem součtu: 4 možnosti na první místo,  3 možnosti na druhé místo, \dots, 1 možnost na poslední čtvrté místo. Tedy $4+3+2+1 = \mathbf{\underline{10}}$
			\feedback Bohužel, v tomto případě nemůžeme použít pravidlo součinu.
			\correct Pravidlem součinu: 4 možnosti na první místo,  3 možnosti na druhé místo, \dots, 1 možnost na poslední čtvrté místo. Tedy $4\cdot 3\cdot 2\cdot 1 = \mathbf{\underline{24}}$
			\feedback Správně! Můžeš pokračovat na \qref[další otázku]{o6}.
			\incorrect Pravidlem součinu: 4 možnosti na první místo,  4 možnosti na druhé místo, \dots, 4 možnosti na čtvrté místo. Tedy $4\cdot 4\cdot 4\cdot 4 = \mathbf{\underline{4^4}}$
			\feedback Uvědom si, že pokud jednu osobu posadíš na jedno místo, už ho nemůžeš posadit na místo další.
			\correct $P(4)= \mathbf{\underline{4!}}$
			\feedback Správně! Můžeš pokračovat na \qref[další otázku]{o6}.
		\end{choice}
	\end{question}
	 
	 \begin{question} \label{o6} %7.1-74a)/85
	 	Kolik existuje všech možných způsobů seřazení osmi různých knih v knihovně vedle sebe?
	 	\answer[integer]{40320}
	 	\whenRight Paráda, permutacím přicházíš na kloub. \qref[Dál]{o7}.
	 	\whenWrong Můžeš využít pravidlo součinu nebo \dref[permutace]{d2}.
	 \end{question}
	 
	 \begin{question} \label{o7} %7.1-74c)/85 upraveno, aby dávalo smysl
	 	Kolik bude existovat způsobů seřazení osmi knih, jestliže u 3 svazků, které mají být vedle sebe, nezáleží na pořadí?\\Výsledek zapište formátem $P(a).P(b), kde \, a \geq b$.
	 	\answer{P(6).P(3)}
	 	\whenRight Dobrá práce, už máš skoro hotovo. \qref[poslední otázka]{o8}.
	 	\whenWrong U tohoto příkladu je důležité si uvědomit, že nejdřív uděláme \dref[permutaci]{d2} knih, které mají být vedle sebe a poté děláme \dref[permutaci]{d2} všech knih, ale pozor, ty knihy, které mají být u sebe už promíchat nesmíme. Nakonec použij jedno ze základních kombinatorických pravidel. 
	 \end{question}
	
	 \begin{question} \label{o8} %7.1-85b)/86
	 	Řešte v oboru $\mathbb{N}$ rovnici s neznámou $n$. \[P(n+1)=30\cdot P(n-1).\]
	 	\answer[integer]{5}
	 	\whenRight Paráda! Už máš hotovo :).
	 	\whenWrong Zkus si rovnici rozepsat pomocí vzorečku pro \dref[permutace]{d2} a už budeš mít hotovo :).
	 \end{question}

\end{document}

























